\documentclass{article}

\usepackage[utf8]{inputenc}
\usepackage[english]{babel}
\usepackage{physics}

\title{Philosophiae Naturalis Principia Mathematica}
\author{Auctore Isaaco Newtono}
\date{MDCLXXXVII}

\begin{document}
\maketitle

\section{Introduzione}
Le leggi della meccanica sono tre, e descrivono il moto di un punto materiale in un campo di forze.

\section{La prima legge}
Un punto materiale non soggetto a forze si muove di moto rettilineo uniforme:
\begin{equation}
	\vb a = 0.
\end{equation}

\section{La seconda legge}
L'accelerazione di un corpo \`e proporzionale alla forza applicata su di esso, secondo la massa:
\begin{equation}
	\vb F = m \vb a.
\end{equation}

\section{La terza legge}
\end{document}
